\documentclass[a4paper,12pt,oneside]{book}

%-------------------------------Start of the Preable------------------------------------------------
\usepackage[english]{babel}
\usepackage{blindtext}
\usepackage{wrapfig}
\usepackage{multirow}
%packagr for hyperlinks
\usepackage{hyperref}
\hypersetup{
	colorlinks=true,
	linkcolor=blue,
	filecolor=magenta,      
	urlcolor=cyan,
}

\urlstyle{same}
%use of package fancy header
\usepackage{fancyhdr}
\setlength\headheight{26pt}
\fancyhf{}
%\rhead{\includegraphics[width=1cm]{logo}}
\lhead{\rightmark}
\rhead{\includegraphics[width=1cm]{logo}}
\fancyfoot[RE, RO]{\thepage}
\fancyfoot[CE, CO]{\href{http://www.e-yantra.org}{www.e-yantra.org}}

\pagestyle{fancy}

%use of package for section title formatting
\usepackage{titlesec}
\titleformat{\chapter}
{\Large\bfseries} % format
{}                % label
{0pt}             % sep
{\Huge}           % before-code

\titlespacing*{\chapter}{0pt}{-50pt}{50pt}

%use of package tcolorbox for colorful textbox
\usepackage[most]{tcolorbox}
\tcbset{colback=cyan!5!white,colframe=cyan!75!black,halign title = flush center}

\newtcolorbox{mybox}[1]{colback=cyan!5!white,
	colframe=cyan!75!black,fonttitle=\bfseries,
	title=\textbf{\Large{#1}}}

%use of package marginnote for notes in margin
\usepackage{marginnote}

%use of packgage watermark for pages
%\usepackage{draftwatermark}
%\SetWatermarkText{\includegraphics{logo}}
\usepackage[scale=2,opacity=0.1,angle=0]{background}
\backgroundsetup{
	contents={\includegraphics{logo}}
}

%use of newcommand for keywords color
\usepackage{xcolor}
\newcommand{\keyword}[1]{\textcolor{red}{\textbf{#1}}}

%package for inserting pictures
\usepackage{graphicx}

%package for highlighting
\usepackage{color,soul}

%new command for table
\newcommand{\head}[1]{\textnormal{\textbf{#1}}}


%----------------------End of the Preamble---------------------------------------

\newcommand\tab[1][1cm]{\hspace*{#1}}

\begin{document}
	
	%---------------------Title Page------------------------------------------------
	\begin{titlepage}
		\raggedright
		{\Large eYSIP2017\\[1cm]}
		{\Huge\scshape Robotic Arm \\[.1in]}
		\vfill
		\begin{flushright}
			{\large Aditya Gaddipati \\}
			{\large Arjun Sadananda \\}
			{\large Simranjeet, Lohit \\}
			{\large Duration of Internship: 22/05/2017-07/07/2017 \\}
		\end{flushright}
		
		{\itshape 2017, e-Yantra Publication}
	\end{titlepage}
	%-------------------------------------------------------------------------------
	
	\chapter[Project Tag]{Robotic Arm}
	\section*{Abstract}
	The motivation behind this project was to \textbf{DESIGN}(Mechanical) a robotic arm and \textbf{PLAN}(Computer Science) and \textbf{CONTROL}(Electronics) the motion of the arm in real time using \textbf{Kinect- Point Cloud}.
	
	\subsection*{Completion status}
	Give details for work/project completed successfully. If work is not
	complete, mention the details till which task is done.
	\begin{itemize}
		\item \textbf{Designed a Robotic Arm} with \textit{6 DOF powered by Dynamixel AX12A}, resembling the dynamics of a \textit{real human arm} from shoulder, elbow to wrist. Complete with a gripper design powered by MicroServo(9g).
		\item \textbf{Controlled} the arm using Arduino Mega (Atmega 2560) through a \textbf{GUI} created using Processing.
		\item \textbf{Interfaced} 6 Dynamixel AX 12A motors with ROS, using Arduino Mega to control the \textbf{flow of packets} between \textbf{dynamixel\_motor} package in ROS and Dynamixel.
		\item Created a \textbf{URDF model} of the arm and mirrored(visualized) the real model movements on the mathematical model using the feedback from Dynamixel on \textbf{RViz}.
		\item Built a \textbf{PCB(Arduino Sheild)} to interface Arduino Mega with Dynamixel via \textit{74LS241 buffer} and power the arm from SMP.
		\item Used \textbf{MoveIt} to generate a trajectory for the robotic arm to move to goal position and send the instructions to the arm.
		\newline
		Not Reliable yet... requires more tests and corrections for reliable motion execution.
		\item Used \textbf{PointCloudLibrary} on roscpp to process the point cloud from the \textbf{Kinect} to get clusters of objects on a table.
		\newline
		Object recognition is not achieved.
	\end{itemize}
	% **************************HARDWARE PARTS*********************
	\section{Hardware parts}
	\begin{itemize}
		\item 6 Dynamixel AX 12A 
		\href{http://www.trossenrobotics.com/images/productdownloads/AX-12(English).pdf}{Datasheet},
		\item Arduino Mega (Atmega 2560)
		\href{https://www.arduino.cc/en/uploads/Main/arduino-mega2560_R3-sch.pdf}{Datasheet}
		\item 1 SMP
		\item 1 74LS241 buffer 
		\href{https://www.arduino.cc}{Datasheet}
		\item Connection diagram
	\end{itemize}
	
	% ***********************SOFTWARE USED*****************
	\section{Software used}
	\begin{itemize}
		\item Fusion 360: version
		\href{https://www.autodesk.com/products/fusion-360/overview}{Official Website} 
		\item Cura ver 2.5.0
		\href{https://ultimaker.com/en/products/cura-software}{Official Website}
		\item ROS Indigo
		\href{https://www.ros.org}{Official Website}
		\newline
		\tab Packages:
		\newline
		\tab \tab \href{http://wiki.ros.org/dynamixel_motor}{dynamixel\_motor}
		\newline
		\tab \tab \href{http://moveit.ros.org/}{MoveIt}
		\newline
		\tab \tab \href{http://wiki.ros.org/freenect_launch}{freenect\_launch}
		\item Arduino IDE
		\href{https://www.arduino.cc}{Official Website}
		\item Processing IDE
		\href{https://www.processing.org}{Official Website}
		\item Library: 
		\href{http://pointclouds.org}{PointCloudLibrary}
	\end{itemize}
	
	% **********************ASSEMBLY OF HARDWARE*********************
	\section{Assembly of hardware}

	\begin{tabular} { p{210pt} p{144pt} }
		Circuit diagram and Steps of assembly of hardware with pictures for each step&
		\multirow{3}{*} {\includegraphics[width=144pt]{circuit.jpg}} \\
		\subsection*{Circuit Diagram}
		Circuit schematic from Eagle
		\subsection*{Arm Assembly}
		Assembling Robotic Arm is self explanatory.
	\end{tabular}
	
	
	\pagebreak
	
	\subsection*{Parts}
	\begin{tabular} { p{118pt}  p{118pt}  p{118pt} }
		\textbf{Shoulder} & \textbf{Elbow} & \textbf{Wrist and Grabber} \\
		\includegraphics[width=118pt]{shoulder.jpg}&
		\includegraphics[width=118pt]{elbow.jpg}&
		\includegraphics[width=118pt]{wrist.jpg}
	\end{tabular}
	
	% ****************SOFTWARE AND CODE**********************
	
	\section{Software and Code}
	\href{http://www.github.com}{Github link} for the repository of code
	\begin{itemize}
		\item dxl\_pc\_bridge.ino: Arduino Code for controlling flow of packets.
		\item dxl\_controller.ino: Arduino code for controlling the robotic arm.
		\item dxl\_controller.pde: Processing code for the processing GUI.
		\item control\_manager.launch: Starts Controller\_Manager
		\item robot\_state.launch: Publishes current state of the robot.
		\item start\_meta\_controller.launch: Starts joint trajectory action controller.
		\item robotic\_arm\_bringup\_rviz.launch: Starts MoveIt!
		\item Object Detection Directory: roscpp code to subscribe to kinect raw point cloud and publishes processed point cloud
		\begin{itemize}
			\item VoxelGrid Downsampling
			\item RANSAC Plane Detection
			\item Conditional Removal
			\item Euclidean Clustering
		\end{itemize}
	\end{itemize}
	
	\pagebreak
	\section{Use and Demo}
	\subsection*{Demo}
	\href{http://www.youtube.com}{Youtube Link} of demonstration video
	\subsection*{Full Assembly}
	
	\begin{center}
	\includegraphics[width=.9\linewidth]{full_assembly.jpg}
\end{center}
	
	
	\section{Future Work}
	\begin{itemize}
		\item MoveIt:
		Execution of MoveIt must be tested and necessary corrections must be done to achieve consistent results.
		\item Kinect:
		Continue work with pcl.
		Use the cluster along with color information and feed it to machine learning module to achieve object recognition.
		\item Design:
		Modification to design can be done to achieve greater range of motion. Ex: slight change in elbow design can increase downward motion of the fore arm.
	\end{itemize}
	
	\section{Bug report and Challenges}
	\begin{itemize}
		\item sync\_write function in dynamixel\_motor -$>$ dynamixel\_driver -$>$ dynamixel\_io.py needed to be modified to call write for every ID, Since in Arduino Code - dxl\_serial we couldnt handle broadcast packages.
		\item Dynamixel AX 12A ID=3 was replaced with ID=6 and ID=6 with AX12W since it completely stopped responding and AX12A ID=1 - ShoulderRoll stopped giving position feedback. These two issues need to be looked into. Temporarily we replaced these with AX12W motors.
		\item The present design provides ~10 kg.cm load at worst cases, but for smooth movement load must be kept below 1/5th of max load i.e. 1/5*15kg.cm
	\end{itemize}
	
	
	\begin{thebibliography}{li}
		\bibitem{rosbook1}Mastering ROS for Robotics Programming: {by Lentin Joseph}
		\bibitem{rosbook2}ROS by Example Vol2: {by R. Patrick Goebel}
		\bibitem{ros}MoveIt: {http://moveit.ros.org/}
		\bibitem{dxl}ROS: {http://wiki.ros.org/dynamixel\_motor}
		\bibitem{f360}Fusion 360: {Youtube Tutorials by Channel: Autodesk Fusion 360}
		\bibitem{wonderTiger}thingiverse.com:
		{\em  WonderTiger, Gripper Design }
		
	\end{thebibliography}
	
	
\end{document}

